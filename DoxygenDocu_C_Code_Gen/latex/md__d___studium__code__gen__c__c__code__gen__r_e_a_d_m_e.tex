Main Page \{\#\+README.\+md\}

Um das Projekt zu \char`\"{}builden\char`\"{}\+:

Win\+:

In Haupverzeichnis wechseln (dort wo der Ordner C\+\_\+\+Code\+\_\+\+Gen und CMake\+Lists.\+txt und Readme.\+md liegt)

neuen Ordner namens \char`\"{}build\char`\"{} anlegen

in build Ordner wechseln

cmd in dem build Ordner öffnen

\char`\"{}cmake -\/\+DCMAKE\+\_\+\+PREFIX\+\_\+\+PATH=\+C\+:\textbackslash{}\+User\+Data\textbackslash{}z180338\textbackslash{}tools\textbackslash{}xerces-\/c -\/\+G \char`\"{}Min\+GW Makefiles\char`\"{} ..\char`\"{} in cmd eingeben

\char`\"{}cmake -\/-\/build .\char`\"{} eingeben

Die Fertige exe kann dann unter build/\+C\+\_\+\+Code\+\_\+\+Gen/\+Debug gefunden werden

Linux/\+Unix\+:

In Haupverzeichnis wechseln (dort wo der Ordner C\+\_\+\+Code\+\_\+\+Gen und CMake\+Lists.\+txt und Readme.\+md liegt)

neuen Ordner namens \char`\"{}build\char`\"{} anlegen

in build Ordner wechseln

terminal in build ordner öffnen

\char`\"{}cmake ..\char`\"{} in terminal eingeben wenn xerces nicht gefunden wurde dann \char`\"{}cmake -\/\+DCMAKE\+\_\+\+PREFIX\+\_\+\+PATH=\char`\"{}opt/xerces-\/c\char`\"{} ..\char`\"{}

\char`\"{}make\char`\"{} in terminal eingeben

Das fertige Programm kann dann unter /build/\+C\+\_\+\+Code\+\_\+\+Gen gefunden werden

Visual Studio \mbox{\hyperlink{class_code}{Code}} und Cmake\+:

Plugins, Cmake suchen und installieren (von Microsoft)

Plugins, C++ suchen und installieren

VSCode neustarten

Den Haupt Ordner in Visual\+Studio\+Code öffnen

Jetzt sollten von cmake unten links Buttons zum builden, Debugen und runnen sicht bar sein\+:

Hinweis zur generierung der Getter Methoden\+: Da in den Requirements immer der long\+Opt-\/\+Name als name in den Methoden gewünscht war haben wir in unserem code wenn ein \textquotesingle{}-\/\textquotesingle{} wie bei sign-\/per-\/line vorkommt, dieses \textquotesingle{}-\/\textquotesingle{} aus dem namen gelöscht und zu signperline (Bsp.\+: is\+Setsignperline()) gemacht.

Hinweis zur Angabe von zusätzlichen Argumenten\+: in unserem generierten Argument-\/\+Parser werden zusätzliche Argumente ohne Zusätze einfach mit einem Leerzeichen getrennt hinter das Hauptargument geschrieben. Bsp.\+: --out-\/path C\+:\textbackslash{}\+Users\textbackslash{} --sign-\/per-\/line 10 -\/v 